\documentclass[12pt,a4paper]{report}
\usepackage{graphicx} % Para inserção de imagens
\usepackage{amssymb}  % Símbolos matemáticos
\usepackage{amsmath}  % Fórmulas matemáticas
\usepackage{pgfplots} % Gráficos
\usepackage[portuguese]{babel}
\usepackage[utf8]{inputenc}
\usepackage[T1]{fontenc}
\usepackage{indentfirst} % Recuo no início do parágrafo
\usepackage{geometry}   % Margens
\usepackage{tocloft}    % Sumário customizado
\usepackage{float}
\usepackage{wrapfig}
\usepackage{subcaption}


% Configuração das margens
\geometry{a4paper,left=3cm,right=2cm,top=3cm,bottom=2cm}

% Configuração do sumário
\renewcommand{\cftchapfont}{\bfseries}
\renewcommand{\cftsecfont}{\normalfont}
\renewcommand{\cftsubsecfont}{\normalfont}
\renewcommand{\cftchappagefont}{\normalfont}

\begin{document}

% CAPA
\begin{titlepage}
    \begin{center}
        \includegraphics[width=5.5cm]{logo.png}\\[0.5cm] % Substitua pelo nome do arquivo
        \textbf{
        {\large INSTITUTO FEDERAL DE EDUCAÇÃO,CIÊNCIAS} \\[0.25cm]
        {\large E TECNOLOGIA DO MARANHÃO} \\[0.25cm]
        {\large CAMPUS AÇAILÂNDIA} \\[2.5cm]
        }

        \textbf{
        {\large  { \scshape Elessandro Silva de Sousa\\ Gustavo Ávalos Ferreira Silva\\ Rodrigo Macedo Oliveira }} \\[3cm]
        }
        
        \textbf{
        {\large  MASSA E CENTRO DE MASSA DE UMA BARRA} 
        }
        
        \vfill
        {\large Açailândia - MA} \\[0.25cm]
        {\large Janeiro de 2025}
    \end{center}
\end{titlepage}

% CONTRA CAPA
\begin{titlepage}
    \begin{center}
        %{\large CENTRO DE MASSA} \\[1cm]

        \textbf{
        {\large { \scshape Elessandro Silva de Sousa\\ Gustavo Ávalos Ferreira Silva\\ Rodrigo Macedo Oliveira } } \\[4cm]
        }

        \textbf{
        {\large MASSA E CENTRO DE MASSA DE UMA BARRA} \\[4cm]
        }
        %{\large Açailândia - MA} \\[0.5cm]
        %{\large Janeiro de 2025}
    \end{center}

        \begin{flushright}
            \parbox{7.9cm}{ % Largura da caixa de texto
            Trabalho apresentado como requisito parcial para 
            a obtenção de nota na disciplina Cálculo 
            diferencial e integral II, ministrada pelo 
            Professor Adão Nascimento dos Passos.
            }
        \end{flushright}

    \begin{center}
        \vfill
        {\large Açailândia - MA} \\[0.25cm]
        {\large Janeiro de 2025}
    \end{center}
\end{titlepage}

% SUMÁRIO
\tableofcontents
\newpage

% LISTA DE FIGURAS
\listoffigures
\newpage

\chapter*{Introdução}
\addcontentsline{toc}{chapter}{Introdução}
\begingroup

\setlength{\parskip}{0.2cm}
O centro de massa de um corpo é um ponto (fictício) sobre o qual é possível considerar que toda a sua massa está concentrada, o que nos permite tratá-lo como um ponto material (ou partícula pontual). A localização desse ponto depende, naturalmente, da massa do corpo, bem como de sua forma geométrica, ou seja, depende da forma como a massa do corpo está distribuída.

Tal definição de centro de massa aplica-se não apenas para um corpo rígido, podendo ser estendida para um sistema formado por N partículas (ou seja, formado por corpos separados). Dessa forma, o centro de massa de um sistema de N partículas é um ponto sobre o qual podemos considerar que a massa total (a soma das massas das N partículas) do sistema está concentrada.

Ao analisarmos uma barra, um objeto aparentemente simples, podemos explorar de forma clara esse conceito. Em uma barra homogênea, o centro de massa coincide com o centro geométrico. No entanto, ao adicionarmos massa em uma das extremidades, o centro de massa se desloca. Essa propriedade é de fundamental importância em diversas áreas, como na engenharia civil, para o cálculo de momentos fletores em vigas, e na aeronáutica, para determinar o centro de gravidade de aeronaves.

Neste estudo, vamos explorar de forma interativa e simples, como a distribuição de massa em uma barra influencia a localização de seu centro de massa, e como esse conceito é fundamental para encontrar o centro de massa de uma barra.
\endgroup

\chapter*{Massa e Centro de Massa de uma Barra}
\addcontentsline{toc}{chapter}{Massa e Centro de Massa de uma Barra}

\begingroup
\setlength{\parskip}{0.2cm}
Inicialmente, vamos descrever o conceito de centro de massa de um sistema constituído por um número finito de partículas, localizadas sobre um eixo $L$, de peso e espessura insignificantes.

Vamos supor que o eixo $L$ esteja na posição horizontal e imaginemos que ele possa girar livremente em torno de um ponto $P$, como se nesse ponto fosse colocado um apoio (ver Figura 1).

%\endgroup
\begin{figure}[H]
    \centering
    \includegraphics[width=0.6\textwidth]{3.jpg} % Substitua "caminho_da_imagem" pelo caminho do arquivo
    \caption{}
    \label{fig:ce}
\end{figure}

Se colocarmos sobre $L$ um objeto de peso $w_1$ a uma distância $d_1$, à direita de $P$, o peso do objeto fará $L$ girar no sentido horário (ver Figura 2 (a)). Colocando um objeto de peso $w_2$, a uma distância $d_2$ à esquerda de $P$, o peso desse objeto fará $L$ girar no sentido anti-horário (ver Figura 2 (b)).

\begin{figure}[h]
    \centering
    \begin{minipage}{0.45\textwidth}
        \centering
        \includegraphics[width=\textwidth]{IMG-20250110-WA0005.jpg}
        \subcaption{}
    \end{minipage}
    \hfill
    \begin{minipage}{0.45\textwidth}
        \centering
        \includegraphics[width=\textwidth]{2.jpg}
        \subcaption{}
    \end{minipage}
    \caption{}
\end{figure}
Colocando simultaneamente os dois objetos sobre $L$ (ver Figura 3), o equilíbrio ocorre quando
\begin{equation}
w_1 d_1 = w_2 d_2.
\end{equation}

\begin{figure}[H]
    \centering
    \includegraphics[width=0.6\textwidth]{4.jpg} % Substitua "caminho_da_imagem" pelo caminho do arquivo
    \caption{}
    \label{fig:ce}
\end{figure}

Este resultado é conhecido como \textbf{Lei da Alavanca} e foi descoberto por Arquimedes. Na prática, podemos constatar isso quando duas crianças se balançam numa gangorra.

Vamos, agora, orientar $L$ e fazê-lo coincidir com o eixo dos $x$ do sistema de coordenadas cartesianas. Se duas partículas de peso $w_1$ e $w_2$ estão localizadas nos pontos $x_1$ e $x_2$, respectivamente (ver Figura 4), podemos reescrever (1) como

\begin{equation*}
w_1 (x_1 - P) = w_2 (P - x_2) \quad \text{ou} \quad w_1 (x_1 - P) + w_2 (x_2 - P) = 0.
\end{equation*}

\begin{figure}[H]
    \centering
    \includegraphics[width=0.6\textwidth]{IMG-20250110-WA0010.jpg} % Substitua "caminho_da_imagem" pelo caminho do arquivo
    \caption{}
    \label{fig:ce}
\end{figure}

Supondo que $n$ partículas de pesos $w_1, w_2, \dots, w_n$ estejam colocadas nos pontos $x_1, x_2, \dots, x_n$, respectivamente, o sistema estará em equilíbrio ao redor de $P$, quando
\begin{equation}
\sum_{i=1}^n w_i (x_i - P) = 0.
\end{equation}

Como o peso de um corpo é dado por $w = mg$, onde $g$ é a aceleração da gravidade e $m$ é a massa do corpo, considerando $g$ constante, podemos reescrever (2) como
\begin{equation*}
\sum_{i=1}^n m_i g (x_i - P) = 0,
\end{equation*}
ou de forma equivalente,
\begin{equation*}
\sum_{i=1}^n m_i (x_i - P) = 0.
\end{equation*}

A soma $\sum\limits_{i=1}^n m_i (x_i - P)$ mede a tendência do sistema girar ao redor do ponto $P$ e é chamada \textit{momento do sistema em relação a $P$}. Quando o momento é positivo, o giro se dá no sentido horário. Quando o momento é negativo, o giro se dá no sentido anti-horário e, obviamente, quando o momento é nulo o sistema está em equilíbrio.

Se o sistema não está em equilíbrio, movendo o ponto $P$, podemos encontrar um ponto $\bar{x}$, de tal forma que ocorra o equilíbrio, isto é, um ponto $\bar{x}$ tal que o momento do sistema em relação a $\bar{x}$ seja nulo. O ponto $\bar{x}$ deve satisfazer


\begin{equation*}
\sum_{i=1}^n m_i (x_i - \bar{x}) = 0.
\end{equation*}

Resolvendo esta equação para $\bar{x}$, obtemos
\begin{equation*}
\sum_{i=1}^n m_i x_i -  \sum_{i=1}^n m_i\, \bar{x} = 0,
\end{equation*}
ou
\begin{equation*}
\bar{x} \sum_{i=1}^n m_i= \sum_{i=1}^n m_i x_i
\end{equation*}
ou ainda
\begin{equation}
    \bar{x} = \frac{\sum\limits_{i=1}^n m_i x_i}{\sum\limits_{i=1}^n m_i}
\end{equation}

O ponto $\bar{x}$ que satisfaz (3) é chamado \textbf{centro de massa do sistema dado}.\\

Sob a hipótese de a aceleração da gravidade ser constante, $\bar{x}$ também é chamado \textbf{centro de gravidade do sistema}.\\
É interessante observar que, na expressão (3), o numerador do lado direito é o momento do sistema em relação à origem e que o denominador é a massa total do sistema\\

Queremos, a seguir, mostrar como a integração pode ser usada para estender essas ideias a um sistema que, ao invés de ser constituído por um número finito de partículas, apresenta uma distribuição contínua de massa.

Consideremos uma barra horizontal rígida, de comprimento $l$. Se a sua densidade linear $\rho$, que é definida como massa por unidade de comprimento, é constante, dizemos que a barra é homogênea. Neste caso, intuitivamente, percebemos que a massa total da barra é dada por $\rho l$ e que o centro de massa deve estar localizado no ponto médio da barra.

Suponhamos, agora, que temos uma barra não homogênea. Localizamos a barra sobre o eixo dos $x$, com as extremidades nos pontos $a$ e $b$, como a Figura 5.

\begin{figure}[H]
    \centering
    \includegraphics[width=0.6\textwidth]{6.jpg} % Substitua "caminho_da_imagem" pelo caminho do arquivo
    \caption{}
    \label{fig:ce}
\end{figure}
Seja $\rho(x)$, $x \in [a, b]$, uma função contínua que representa a densidade linear da barra. Para encontrar a massa total da barra, vamos considerar uma partição $P$ de $[a, b]$, dada pelos pontos
\[
a = x_0 < x_1 < \dots < x_{n-1} < x_i <\cdots < x_n = b.
\]

Sejam $c_i$ um ponto qualquer do intervalo $[x_{i-1}, x_i]$ e $\Delta x_i = x_i - x_{i-1}$. Então, uma aproximação da massa da parte da barra entre $x_{i-1}$ e $x_i$ é dada por:
\[
\Delta m_i = \rho(c_i) \Delta x_i,
\]
e
\begin{equation}
\sum_{i=1}^n \Delta m_i = \sum_{i=1}^n \rho(c_i) \Delta x_i
\end{equation}
constitui uma aproximação da massa total da barra.

Podemos observar que, à medida que $n$ cresce muito e cada $\Delta x_i \to 0$, a soma (4) se aproxima do que intuitivamente entendemos como massa total da barra.




Assim, como (4) é uma soma de Riemann da função contínua $\rho(x)$, podemos definir a massa total da barra como
\begin{equation}
    m = \int_{a}^{b} \rho(x) \, dx 
\end{equation}

Para encontrarmos o centro de massa da barra, precisamos primeiro encontrar o momento da barra em relação à origem.

Procedendo de acordo com as hipóteses e notações anteriores, obtemos que $c_i \Delta m_i$ é uma aproximação do momento em relação à origem, da parte da barra que está entre $x_{i-1}$ e $x_i$, e que
\begin{equation}
    \sum_{i = 1}^{n} c_i \Delta m_i = \sum_{i = 1}^{n} c_i \rho(c_i) \Delta x_i
\end{equation}

é uma aproximação do momento da barra em relação à origem.

Como a soma (6) é uma de Riemann da função contínua $x \rho(x)$, podemos definir o momento da barra em relação à origem como
\begin{equation}
    M_0 = \int_{a}^{b} x \rho(x) \, dx
\end{equation}

Então, entendendo a expressão (3) para a barra, obtemos o seu centro de massa $\bar{x}$, que é dado por
\endgroup

\begin{equation}
    \bar{x} = \frac{1}{m} \int_{a}^{b} x \rho(x) \, dx \tag{8}
\end{equation}

\chapter*{Exemplos}
\addcontentsline{toc}{chapter}{Exemplos}

\subsection*{Exemplo (i):}
Usando (8), verificar que o centro de massa de uma barra homogênea está no seu ponto médio.


\subsubsection*{Solução} 
Seja $l$ o comprimento da barra e $\rho$ a sua densidade linear. Localizando a barra sobre o eixo dos $x$ com extremidades nos pontos $a$ e $b$, temos:
\begin{align*}
    m &= \int_{a}^{b} \rho \, dx \\
      &= \rho \int_a^b dx \\
      &= \rho \, (b - a) \\
      &= \rho l \quad \text{(unidades de massa);}
\end{align*}

\begin{align*}
    \bar{x} &= \frac{1}{m} \int_{a}^{b} x \rho \, dx \\[0.2cm]
            &= \frac{\rho}{\rho l} \int_a^b x \, dx \\[0.2cm]
            &= \frac{1}{l} \int_a^b x \, dx \\[0.2cm]
            &= \frac{1}{l}\, \frac{x^2}{2}\Big|_a^b \\[0.2cm]
            &= \frac{1}{l} \frac{1}{2} (b^2 - a^2) \\[0.2cm]
            &= \frac{1}{2l} (b^2 - a^2)
\end{align*}

\[
    = \frac{1}{2l} (b - a)\cdot (b + a)
\]
\noindent
Como $b - a = l$, temos:
\[
\bar{x} = \frac{b - a}{2}
\]
ou seja, $\bar{x}$ est\'a sobre o ponto m\'edio da barra.\\

\noindent
Neste exemplo, fica claro que a localiza\c{c}\~ao do centro de massa em rela\c{c}\~ao \`a barra n\~ao depende da posi\c{c}\~ao da barra em rela\c{c}\~ao \`a origem. Na pr\'atica, podemos sempre escolher a posi\c{c}\~ao mais conveniente de forma a facilitar os c\'alculos.

\subsection*{Exemplo (ii):}

\noindent
Uma barra mede 6 m de comprimento. A densidade linear em um ponto qualquer da barra \'e proporcional \`a dist\^ancia desse ponto a um ponto $q$, que est\'a sobre o prolongamento da linha da barra, a uma dist\^ancia de 3 m da mesma. Sabendo que na extremidade mais pr\'oxima a $q$, a densidade linear \'e $1$ kg/m, determinar a massa e o centro de massa da barra.

\subsubsection*{Solu\c{c}\~ao:}

A Figura 6 mostra a barra localizada sobre o eixo dos $x$:

\begin{figure}[H]
    \centering
    \includegraphics[width=0.6\textwidth]{7.jpg} % Substitua "caminho_da_imagem" pelo caminho do arquivo
    \caption{}
    \label{fig:ce}
\end{figure}

\noindent
A dist\^ancia de um ponto $x$ da barra at\'e $q$ \'e dada por:


\begin{align*}
    d &= x - (-3)\\ 
    &= x + 3.
\end{align*}



\noindent
Como a densidade \'e proporcional \`a dist\^ancia $d$, temos:
\[
\rho(x) = k(x + 3),
\]
onde $k$ \'e uma constante de proporcionalidade.

\noindent
Como $\rho(0) = 1$ kg/m, substituindo na express\~ao anterior, vem:
\[
1 = k(0 + 3) \quad \Rightarrow \quad k = \frac{1}{3}.
\]

\noindent
Portanto:
\[
\rho(x) = \frac{1}{3}(x + 3), \quad \forall x \in [0, 6].
\]

\noindent
A massa da barra \'e dada por:
\[
m = \int_{0}^{6} \rho(x) \, dx = \int_{0}^{6} \frac{1}{3}(x + 3) \, dx.
\]

\noindent
Resolvendo a integral:
\[
m = \frac{1}{3} \int_{0}^{6} (x + 3) \, dx = \frac{1}{3} \left( \frac{x^2}{2} + 3x \right)\Bigg|_{0}^{6}
\]

\noindent
Substituindo os limites:
\[
m = \frac{1}{3} \left[ \frac{6^2}{2} + 3(6) - \left( \frac{0^2}{2} + 3(0) \right) \right] = \frac{1}{3} \left[ 18 + 18 \right] = 12 \, \text{kg}.
\]

\subsubsection*{Centro de Massa}

\noindent
O centro de massa da barra \'e dado por:
\[
\bar{x} = \frac{1}{m} \int_{0}^{6} x \rho(x) \, dx = \frac{1}{12} \int_{0}^{6} x \left( \frac{1}{3}(x + 3) \right) \, dx.
\]

\noindent
Simplificando:
\[
\bar{x} = \frac{1}{12} \cdot \frac{1}{3} \int_{0}^{6} x(x + 3) \, dx = \frac{1}{36} \int_{0}^{6} (x^2 + 3x) \, dx.
\]

\noindent
Resolvendo a integral:
\[
\int_{0}^{6} (x^2 + 3x) \, dx = \left( \frac{x^3}{3} + \frac{3x^2}{2} \right)\Bigg|_{0}^{6}
\]

\noindent
Substituindo os limites:
\[
\int_{0}^{6} (x^2 + 3x) \, dx = \left[ \frac{6^3}{3} + \frac{3(6^2)}{2} \right] - \left[ \frac{0^3}{3} + \frac{3(0^2)}{2} \right] = 72 + 54 = 126.
\]

\noindent
Portanto:
\[
\bar{x} = \frac{1}{36} \cdot 126 = 3.5 \, \text{m}.
\]

\noindent
Assim, a massa da barra \'e $12 \, \text{kg}$ e o centro de massa est\'a localizado a $3.5 \, \text{m}$ da extremidade esquerda.

\subsection*{Exemplo (iii):}

Determinar o centro de massa de uma barra de $5 \, \text{m}$ de comprimento, sabendo que num ponto $q$, que dista $1 \, \text{m}$ de uma das extremidades, a densidade é $2 \, \text{kg/m}$ e que nos demais pontos ela é dada por $(2 + d) \, \text{kg/m}$, onde $d$ é a distância até o ponto $q$.

\subsubsection*{Solução}

Localizamos a barra sobre o eixo dos $x$ como mostra a Figura 7:

\begin{figure}[H]
    \centering
    \includegraphics[width=0.6\textwidth]{8.jpg} % Substitua "caminho_da_imagem" pelo caminho do arquivo
    \caption{}
    \label{fig:ce}
\end{figure}

Podemos expressar a densidade da barra pela função:
\[
\rho(x) =
\begin{cases}
2, & x = 4, \\
2 + (4 - x) = 6 - x, & 0 \leq x < 4, \\
2 + (x - 4) = x - 2, & 4 < x \leq 5.
\end{cases}
\]

A massa da barra é dada por:
\[
m = \int_{0}^{5} \rho(x) \, dx = \int_{0}^{4} (6 - x) \, dx + \int_{4}^{5} (x - 2) \, dx.
\]

Resolvendo as integrais:
\[
\int_{0}^{4} (6 - x) \, dx = \left( 6x - \frac{x^2}{2} \right)\Bigg|_{0}^{4} = (24 - 8) = 16,
\]
\[
\int_{4}^{5} (x - 2) \, dx = \left(\frac{x^2}{2} - 2x\right)\Bigg|_{4}^{5} = \left(\frac{25}{2} - 10\right) - \left(\frac{16}{2} - 8\right) = 2.5.
\]

Somando:
\[
m = 16 + 2.5 = 18.5 \, \text{kg}.
\]

O centro de massa é dado por:
\[
\bar{x} = \frac{1}{m} \int_{0}^{5} x \rho(x) \, dx.
\]

Resolvendo com os valores apropriados, podemos calcular o centro de massa.
\newpage
\chapter*{Referências}
\addcontentsline{toc}{chapter}{Referências}

\textbf{Halliday}, D.; Resnick, R.; Walker, J. Fundamentos de Física.
\end{document}
